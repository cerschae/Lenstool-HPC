\input macro.tex
\chapter{Input File}

Here is a rapid description of each keywords that appear in the
\textit{inputfile}. Keywords are of 2 types: first identifier and
second identifier. First identifier are more general and deal with a specific 
part of the program (like computation of the critical and caustic curves).
Second identifiers are specialized for each first identifier, and define
some constant or file.

\section{First identifiers}

\begin{description}

\item{{\bf runmode \ }} this identifier is the most important one, and determine what the
program will do. (compulsory)

\item{{\bf grille \ }} defines some parameters such has the number of potential mode,
the total 
number of potential mode that are going to be test, the grid mode, and the
number of rows and columns in the grid. (compulsory)

\item{{\bf potential \ } or {\bf potentiel}}
under this identifier is defined one mode of the
gravitational potential. One can define a global potential with many modes,
for each mode a first identifier "potential" must be defined.

\item{{\bf limit \ }} under this identifier are defined the
constraints on the potential
(more precisely on one mode). This identifier has to follow the identifier
of the corresponding "potential" mode. (used only in the invert runmode)

\item{{\bf potfile \ }} under this identifier are defined the 
default parameters for all the galaxy scale mass components that account
for perturbations to the cluster potential by the galaxies. 

\item{{\bf cline \ }} under this identifier are defined the 
parameters to compute the critical and the caustic lines.

\item{{\bf cosmologie \ }or {\bf cosmology \ }} under this
identifier are defined the cosmological parameters
$\Omega_0$, $\lambda$ and $H_0$.

\item{{\bf champ \ }} under this identifier is defined the size of the field used in some
calculations such as the dimension of the grid for the inversion of the lens
equation.

\item{{\bf grande \ }}  under this identifier is defined the way to represent the
computed deformation of objects.

\item{{\bf observ \ }}  under this identifier is defined the different noise that can
be add to a gravitational image, such as seeing or Poisson Noise.

\item{{\bf source \ }}  under this identifier are defined some 
characteristics of sources when a random drawing is done.

\item{{\bf image \ }}  under this identifier are defined some characteristics of images,
multiple images or arclets.

\item{{\bf cleanlens \ }}  
under this identifier are defined some parameters to
retrieve the shape of the source knowing a pixel-frame of the image.

\item{{\bf fini \ }} tells the program to stop the reading of the inputfile. (compulsory)

\end{description}

\section{Second identifiers}
\label{sect:secondid}

For each first identifier, we will defined the second identifier, and gives
the default value, and their uses, with an example.

\subsection{runmode}
\label{sect:secondid:runmode}

\textit{this identifier is the most important one, and determine
 what the
program will do.} \\

\begin{description}
\item{{\bf reference \ } \textit{int} \textit{RA} \textit{DEC} }\\
Set the reference point for the system to study. This keyword is used to convert
the relative coordinates used in \lenstool to absolute coordinates used
in some input or output files.\\
If \textit{int}=1, \textit{RA} and \textit{DEC} are in the sexagesimal format $hh:mm:ss$ and $dd:mm:ss$. \\
If \textit{int}=3, \textit{RA} and \textit{DEC} are in degrees.\\

\item{{\bf arclet \ } \textit{int} \textit{filename} \ \ or \ \ 
{\bf image \ } \textit{int} {\sl filename} }\\
\textit{int \ } 0: if false 1: if true.\\
If true the program will read a list of arclet in \textit{filename} find their
sources [ and put them in the output file \textit{source.dat} ] 
and recompute all their images [ and put them in the output file \textit{image.all} ].\\
Moreover the program will create other 'arclets' output files:\\
--\textit{image.dat} where only the computed arclet with deformation
$\tau$ less than {\bf grande large\_dist\ } are indicated (basically, 
\textit{image.dat} does not include giant arcs).\\
--\textit{dist.dat} information about the ellipticity of all the arclets, 
including weakly distorted images and giant arcs. \\
--\textit{sort.dat}  same as {\sl image.all} but
sorted from the most elongated (giant arcs) to the less (arclets).\\

Interest: find counter images.\\

Format: the format of \textit{filename} is describe in Sect. 3.1.1. It is
an ASCII column format of the form:\\
\{ $i$ $x_i$ $y_i$ $a_i$ $b_i$ $\theta_i$ $z_i$ \}\\

Note: $\theta_i$ is 90$^\circ$ relative to the PA definition, $x$ and $y$ are RA
and Dec in decimal degrees.\\
\\
\framebox[\linewidth][l]{%
\begin{minipage}[]{0.98\linewidth}
\emph{VISUALISATION:}\\
PICT: \textit{source.dat}, {\sl image.dat}, {\sl image.all}, {\sl sort.dat}
are 'arclet' style file (see Sect. 3.2.3) and can be visualized with
PICT under the {\bf arclet} qualifier:\\
To visualize  the file \textit{image.all} and {\sl source.dat} one has to write
in the '\textit{input}.in' PICT input file:\\
{\bf
\begin{tabular}{lllll}
arclet&    & & & \\
 &narc     &2& & \\
 &namein   &1&0& image.all\\
 &namein   &2&0& source.dat\\
 &end      & & & \\
frame&    & & & \\
 &dmax     &30.& & \\
 &end      & & & \\
fini&    & & & \\
\end{tabular}
}\\
This will display the ellipses of \textit{image.all} and {\sl source.dat}
in a frame of size [x: -30. +30, y:-30. +30]\\
\end{minipage}
}

\item{{\bf source \ } \textit{int} {\sl filename} }\\
\textit{int \ } 0: if false 1: if true.\\
If true the program will read a list of sources in \textit{filename} 
[ and put them in the output file \textit{source.dat} ] 
and compute all their images
[ and put them in the output file \textit{image.all} ].\\

Moreover the program will create the same 'arclets' output files as for
the {\bf arclet} preceding sub-qualifier:\\
-- \textit{image.dat} where only the computed arclet with deformation
$\tau$ less than {\bf grande large\_dist\ } are indicated.\\
\textit{dist.dat} information about the ellipticity of all the arclet 
sorted from the most elongated to the less.\\
-- \textit{sort.dat}  same as {\sl image.all} but
sorted from the most elongated to the less.\\

Interest: usually used to show typical image configurations.\\

PICT:  see previous identifier\\

Format:  see previous identifier\\
Note: both {\bf arclet/image} and {\bf source} identifiers
 can be used at the same time.\\


\item{{\bf time \ } \textit{int1} {\sl int2} {\sl float} {\sl filename}}\\
\textit{int1 \ } 0: if false 1: if true.\\
If true will compute
for the redshift \textit{float} a pixel-frame
(\textit{int2} $\times$ {\sl int2}) of the (relative) arrival time (in year)
of each pixel of the image plane (area defined by {\bf frame \ }).\\
Results are written in the pixel-frame file \textit{filename}\\
The format of \textit{filename} is the 'ipx' simple pixel-frame format (format 2
for PICT).\\ 

Note: The arrival time is defined by:
$$
	\tau_a(\vec{ \xi^I},z^S)= {1\over c}{D_{LS}D_{OL}\over D_{OS}}
			\left( \|\nabla\varphi(\vec{ \xi^I},z^S)\|^2
				- \varphi(\vec{ \xi^I},z^S) \right)
$$

$\varphi$ is the lens-normalized projected potential:
$$
\varphi={2\over c^2} {D_{LS} D_{OL} \over D_{OS}} \phi \ ,
$$
where $\phi$ is the Newtonian projected potential:\\
$$
\phi =  {1 \over D_{OL}^2 }\int{ \Phi^{3D} dl}\ .
$$
The (relative) arrival time surface corresponds to the time of arrival
at the Image Plane of a flash that left at the same time the Source Plane.
It it absolute in the sense we have substract the mean travel time between
the two plane. To find out the time-delay between two images of the same source
one has to know the position of the images and then compute the difference
between the two corresponding arrival time.\\


PICT: The following example of a PICT inputfile allow to display
the pixel-frame \textit{filename} with a gray lut starting at zgmin=-10 (white)
ending at zgmax=10 (black) 
in a frame of size [x: -20. +20, y:-20. +20]\\
The position of the pixel-frame is automatically scaled (scale= 1).\\

{\bf
\begin{tabular}{lllll}
contour&         & &  \\
        &filein  &1&\textit{filename}\\
        &format  &2& \\
        &scale   &1&  \\
        &zgmin   &-10&  \\
        &zgmax   &10&  \\
        &end     & & \\
frame&         & &  \\
        &dmax    &20.&  \\
        &end     & & \\
fini&     & & \\
\end{tabular}
}\\
When the pixel-frame of the time delay is displayed, one can then get the
value of the time-delay with the cursor using the PICT command (c)oord.\\

\item{{\bf ampli \ } \textit{int1} {\sl int2} {\sl float} {\sl filename}}\\
\textit{int1 \ } 0: if false 1: if true.\\
Same as {\bf time \ } but compute the amplification $\mu^-1$ in the image plane.\\
If \textit{int1}= 1, compute $\mu^-1$, if \textit{int1}=2, compute its absolute value and if \textit{int1}=3, compute its absolute value in magnitude.\\
If \textit{int1}=5, compute the convergence map in the image plane.
If \textit{int1}=6, compute the shear map in the image plane.\\
If \textit{int1}=-1, compute the absolute value of $\mu^-1$ in the source plane considering every images (prototype).\\

The amplification is defined by:\\
$$
\mu =  \left( \left(1- \kappa \right)^2 -  \gamma^2 \right)
^{-1}\ , 
$$
with $\kappa$ and $\gamma$ are the convergence and the shear
respectively. They are defined by:\\
$$
2 \kappa=\nabla^2 \varphi \ ,
$$
and
$$
\gamma^2={1 \over 4} \Bigr(\partial_{xx} \varphi-\partial_{yy} \varphi
\Bigl)^2+
\Bigr(\partial_{xy} \varphi \Bigl)^2 \ .
$$ 

PICT: see {\bf time}\\

\item{{\bf poten \ } \textit{int1} {\sl int2} {\sl float} {\sl filename}}\\
\textit{int1 \ } 0: if false 1,2: if true.\\
\textit{int2 \ } size of the square grid.\\
\textit{float \ } redshift of the source plane ($z_S$).\\
Same as {\bf time \ } but compute the relative (\textit{int1}= 1) or
absolute (\textit{int1}= 2) projected potential.

The relative projected potential (\textit{int1 \ }= 1)
is defined by
\\
$$
\varphi(\vec{ \xi^I},z^S) =
{2 \over c^2}D_{OL}{D_{LS} \over D_{OS}} \phi(\vec{ \xi^I})
$$
Where the absolute projected potential (\textit{int1 \ }= 2) is: 
$ \phi(\vec{ \xi^I}) $\\
Because it is absolute it does not depend of the redshift $z_s$ (\textit{float}).

PICT: see {\bf time}\\


\item{{\bf mass \ } \textit{int1} \textit{int2} \textit{float} \textit{filename} }\\
\textit{int1 \ } 0: if false 1,2: if true.\\
\textit{int2 \ } size of the square grid.\\
\textit{float \ } redshift of the source plane ($z_S$).\\
Same as {\bf time \ } but compute the relative (\textit{int1}= 1) the
absolute (\textit{int1}= 2 and 4) projected mass-density, or the integrated
(\textit{int1}= 3) projected mass-density.\\

The relative projected mass-density (\textit{int1 \ }= 1)
(called also convergence) is determined by
\\
$$
\kappa(\vec{ \xi^I},z^S)={\nabla^2\varphi(\vec{ \xi^I},z^S) \over 2} =
{\Sigma(\vec{ \xi^I},z^S) \over \Sigma_{crit}}\ .
$$
The critical density is defined by:\\
$$
\Sigma_{crit}(z^S)= {c^2 \over 4\pi G}{D_{OS} \over D_{LS}D_{OL}}
$$
The absolute projected mass-density (\textit{int1}= 2 and 4) is determined by:\\
\begin{center}
\begin{tabular}{cc}
$ \Sigma(\vec{ \xi^I})= \Sigma_{crit}{\nabla^2\varphi \over 2}=
{\nabla^2_{\vec{ \xi^I}}\phi(\vec{ \xi^I}) \over 4 \pi G} $
 & 
\begin{tabular}{c}
$in \ g/cm^2\ (int=2)$ \\
$in \ 10^{12}M_{\odot}/kpc^2\ (int=4)$\\
\end{tabular}
\end{tabular}
\end{center}

Because it is absolute, it does not depend of the redshift $z_s$ (\textit{float}).

The integrated projected mass-density (\textit{int1 \ }= 3)  is determined by:\\
$$
M(\vec{ \xi^I})= {\nabla^2_{\vec{ \xi^I}}\phi(\vec{ \xi^I}) \over 4 \pi G}
S_{pixel}\ \ \ \ \ \ \ (in \ 10^{12} M_{\odot}/pixel)
$$
Obviously, it depends on the pixel size. As for the \textbf{time} function, the area  covered by the image is defined  in the \textbf{frame} section.

In Bayesian optimisation mode, you can get the projected error mass-density (\textit{int1 \ }=5) in $10^{12} M_{\odot}/pixel$. The map size in arcsec is defined with the \verb+dmax+ keyword in the \verb+champ+ section. In the rectangular field case, the image size is (X,Y) = (scaling*\textit{int2}, \textit{int2}).

PICT: see {\bf time}\\


\item{{\bf shear \ } \textit{int1} {\sl int2} {\sl float} {\sl filename}}\\
\textit{int1 \ } 0: if false, true otherwise.\\
Same as {\bf time \ } but compute :\\
{\it int1=1} the shear $\gamma$ defined by:\\
$$
\gamma=\sqrt{{1 \over 4} \Bigr(\partial_{xx} \varphi-\partial_{yy} \varphi
\Bigl)^2 +
\Bigr(\partial_{xy} \varphi \Bigl)^2} \ .
$$
{\it int1=2} the ellipticity $\epsilon$ defined by:\\
$$
\epsilon=\frac{q^2 - 1}{q^2 +1} \ ,
$$
\noindent where q is the ratio of the amplification matrix eigenvalues
$\lambda_1 / \lambda_2$.\\

If {\it int1 < 0}, the behavior is the same but for pixels considered
in the source plane.\\

\item{{\bf shearfield \ } \textit{int} {\sl float} {\sl filename} {\sl int2}}\\
\textit{int \ } 0: if false 1,2: if true.\\
If true will compute
for the redshift \textit{float} the shear field
at [\textit{int2}$\times$\textit{int2}] points of the Image Plane (area defined by {\bf frame \ }).\\
Results are written in the 'arclet'-type  file \textit{filename}\\
If \textit{int\ }= 1 the size of the ticks correspond to the induced ellipticity
by the mass distribution.\\
If \textit{int\ }= 2 the ticks show only the polarization of the field.\\

Note: Do not mix the identifiers {\bf shear} and {\bf shearfield}.
{\bf shear} is just a pixel-frame of the intensity of the shear with no
indication of the orientation of the shear. {\bf shearfield} on the 
contrary will give you the orientation of the shear and its intensity
but only in \textit{int2}x\textit{int2} points.\\
Default \textit{int2}=25. \\
PICT: The following example of a PICT inputfile allow to display
the shearfield \textit{filename} 
in a frame of size [x: -20. +20, y:-20. +20]\\

{\bf
\begin{tabular}{lllll}
arclet&    & & & \\
 &narc     &1& & \\
 &namein   &1&0& \textit{filename}\\
 &end      & & & \\
frame&    & & & \\
 &dmax     &20.& & \\
 &end      & & & \\
fini&    & & & \\
\end{tabular}
}\\

\item{{\bf study \ } \textit{int} {\sl filename} }\\
\textit{int \ } 0: if false 1: if true.\\
The purpose of {\bf study \ }is a statitistical analysis of the arclets
properties to infer the probable redshift of sources.\\
If true the program will read a list of arclet in \textit{filename} 
and computes for different redshift the ellipticity, size and orientation
of the sources. It will also give the z0- zm- zmin zm+ z0+ (file z.dat)
as defined in Kneib et al. 1994.\\
The format of \textit{filename} is exactly the same as the one of {\bf arclet}.\\
This program will create four output files:\\
-- \textit{ess.dat}\ : It is the record of the variation of the ellipiticy
with the redshift for all the arclets of \textit{filename}. {\sl ess.dat} is
an ASCII file, that has the following format:\\
\{ $i$ $z_{ij}$ $dr_{ij}$ $\tau^S_{ij}$ $\varepsilon^S_{ij}$
$\theta_{ij}$ $n_{ij}$ $\Delta_{ij}$ $ez_i$ \}\\
where $i$ is the index of the arclet, $z_{ij}$ the redshift at step $j$=1,N ,
$dr_{ij}$ the cosmological ratio $D_{LS}/D_{OS}$, $\tau^S_{ij}$ the
deformation, $\varepsilon^S_{ij}$ the ellipticity, $\theta_{ij}$
the orientation ($90^{\circ}$ relative to PA), $n_{ij}$ the multiplicity, $\Delta_{ij}$ a value that
is equal to zero when $\tau^S_{ij}$ is minimal, $ez_i$ the estimated
most probable redshift.\\
-- \textit{z.dat}\ : is a synthetic file with th following format:\\
\{ $i$ $z0-_i$ $zm-_i$ $zmin_i$ $zm+_i$ $z0+_i$ $(a/b)^I_i$ $(a/b)^S_i$ $ez_i$
$\tau^S(ez_i)$ \}\\
where $i$ is the index of the arclet, $zmin_i$ is the true most probable
redshift, $z0-_i$ $zm-_i$ $zm+_i$ $z0+_i$ the errors on $zmin_i$,
$(a/b)^I_i$ is the axis ratio of the arclet, $(a/b)^S_i$ is the axis ratio
of the source at $zmin_i$, $ez_i$ is the estimated most probable redshift
(by looking at the zero of $\Delta_{ij}$, $\tau^S(ez_i)$ is the deformation at
$ez_i$.\\
-- \textit{source\_T.dat}\ : is a extended object (ellipse) file with the format:\\
\{ $i$ $x^S_i$ $y^S_i$ $a^S_i$ $b^S_i$ $\theta^S_i$ $zmin_i$ $zm-_i$ $zm+_i$
$(a/b)^S_i$ $\mu_i$\}\\
where $i$ is the index of the arclet, $x^S_i$ $y^S_i$ the position of the
source, $a^S_i$ $b^S_i$ the major and minor semi-axis of the source, 
$\theta^S_i$ the orientation ($90^{\circ}$ relative to PA) of the source 
at redshift $zmin_i$,
$zm-_i$ $zm+_i$ are the errors on $zmin_i$, $(a/b)^S_i$ is the axis ratio
of the source at $zmin_i$, $\mu_i$ is the amplification of the image 
for a source at $zmin_i$.\\
-- \textit{azT.dat}\ : is an extended arclet file with the format:\\
\{ $i$ $x^I_i$ $y^I_i$ $a^I_i$ $b^I_i$ $\theta^I_i$ $zmin_i$
$\mu_i$ $\tau^S(zmin_i)$\}\\
where $i$ is the index of the arclet, $x^I_i$ $y^I_i$ the position of the
arclet, $a^I_i$ $b^I_i$ the major and minor semi-axis of the arclet,
$\theta^I_i$ the orientation of the arclet ($90^{\circ}$ relative to PA), $zmin_i$ is the true most probable
redshift, $\mu_i$ is the amplification of the image
for a source at $zmin_i$, $\tau^S(zmin_i)$ is the deformation of th source at
$zmin_i$.\\

Note:  \textit{source\_T.dat}\ and {\sl azT.dat}\ can be visualized by PICT
without problems by the same way as \textit{image.dat} or {\sl source.dat}.\\

\item{{\bf imseeing \ } \textit{float \ } }\\
of the arclet assuming that both the profile of the arclet and of the seeing
are Gaussian. It's a very simple and crude correction.\\
Used in the {\bf study} mode, and {\bf inverse arcletstat } mode.\\
Default value is 0 (meaning no seeing correction).\\

\item{{\bf grille \ } \textit{int1} \textit{int2} \textit{float} }\\
\textit{int1} 0: if false 1,2: if true.\\
If true, will create a grid of \textit{int2} points. \\
If \textit{int1}= 1, it considers this grid as the Source Plane at the redshift of \textit{float}, and compute the corresponding grid in the Image Plane.\\
If \textit{int1}= 2, it considers this grid as the Image Plane and
compute the corresponding grid in the Source Plane at the
redshift of \textit{float}.\\
The gird coordinates either in the source or in the image planes are defined in the \textbf{frame} section by the \textit{dmax} or by the set of keywords (\textit{xmin}, \textit{xmax}, \textit{ymin}, \textit{ymax}).\\
Results are in the files: gi1.dat (image vertical grid)
gi2.dat (image horizontal grid) and gs1.dat (source vertical grid)
gs2.dat (source horizontal grid).\\
All these file have the following format:\\
\{$j$ $x_i$ $y_i$ \}\\
where $j$ is an index, $x_i$ $y_i$ are the pixel coordinates.\\

PICT: here is an example of PICT inputfile that draw the image grid:\\
{\bf
\begin{tabular}{lllll}
curve&         & & & \\
        &nfile   &2& & \\
        &namein  &1&0&gi1.dat\\
        &column  &1&3& \\
        &        &2&3& \\
        &namein  &2&0&gi2.dat\\
        &column  &2&3& \\
        &        &2&3& \\
        &end     & & & \\
fini&     & & & \\
\end{tabular}
}\\

The grid may also be displayed on ds9 by using the \textit{grid} perl script.\\

\item{{\bf inverse \ } \textit{int1} \textit{float1} [\textit{float2}] }\\
\textit{int1 \ } 0: if false $>$ 1: if true.\\
If true, will enter the optimization mode.\\
If \textit{int1}=1, the optimisation method is the parabolic method.\\
If \textit{int1}=2, the optimisation looks for the galaxy scale parameters sigma and cut radius that give the best lens model. According to the gridding stated in the \verb+potfile+ section, it runs over the grid and computes the best $\chi^2$ in each node with the parabolic optimisation method. (see \verb+potfile+ Section). \\
If \textit{int1}=3, the optimisation method is the bayesian method. This optimisation method is very slow but is less sensible to local $\chi^2$ minima than the parabolic method.\\
If \textit{int1}=4, the optimisation method is a maximum likelihood method but based on the BayeSys algorithm. The cooling factor is not limited to 1.\\

For inverse method 1 and 2, \textit{float1} gives the maximum number of iterations for the parabolic method. \\
For inverse method 3, \textit{float1} gives the speed of calculation of the Bayesian optimisation and \textit{float2} sets the number of sampling iterations. As we use $10$ Markov chains at the same time, each iteration produces $10$ samples. The default value is the number of iterations needed to complete the Burn-in phase. The samples are saved in the \verb+bayes.dat+ ASCII file. \\ 
\textit{float1} sets the rate $\delta\lambda$ by which is raised the likelihood at each step of the Markov Chain. At the beginning, $\lambda = 0$ and at the end of the Markov Chain $\lambda = 1$. Default ($\delta\lambda = 0.5$).\\
The optimisation process will create a \verb+best.par+ and a \verb+bestopt.par"+ files. They contain the best model and the best model + optimisation limits respectively. Additionally, useful information related to the optimisatoin are provided in their header. \\


\item{{\bf minchi0 \ } \textit{float} }\\
\textit{float\ } value of the $\chi^2$ at which the optimization program
will stop in the case of a parabolic optimisation. Default is 0, but this  is not dramatic is case of slow convergence or even non-convergence at all. The number of iterations is also controlled from {\bf inverse} qualifier.\\


\item{{\bf prop \ } \textit{int} {\sl float} {\sl filename} }\\
\textit{int \ } 0: if false 1: if true.\\
If true, will compute for the current potential some properties for a Source
Plane at redshift \textit{float}. This include the orientation of the shear, the magnification, convergence, shear, $\tau_{pot}$, etc...\\
Results are put in the files: \textit{filename}.
This is a huge datafile, hence be careful!\\
Moreover it is not advise to use it, because it is not fast. Users are advised
to use the {\bf time}, {\bf mass}, {\bf ampli}, {\bf shear} and
 {\bf shearfield} identifiers (much faster).\\


\item{{\bf pixel \ } \textit{int1} {\sl int2} {\sl filename} }\\
\textit{int1 \ } 0: if false 1: if true.\\
If true, will create a pixel-frame \textit{filename}
of \textit{int2}$\times${\sl int2} pixels
that corresponds of the brightness intensity of all the arc(let)s computed
from the objects defined in the {\bf image/arclet}
or/and the {\bf source} identifiers.\\
If {\bf observ } is set the program will convolve the true image by
a seeing and add Poisson Noise.\\
The purpose of {\bf pixel} is to make realistic images of arc(let)s from
a given projected potential which can be compared directly with real CCD
images of arc(let)s (see for example Kneib et al. 1993, Fig. 1b). 
{\bf pixel} can make images of several arc(let)s 
coming from various sources on the same frame
(Note: this is not the case with the {\bf iso} identifier).\\

\item{{\bf marker \ } \textit{int} {\sl float} {\sl filename} }\\
\textit{int \ } 0: if false 1: if true.\\
If true, will read the points markers in the file \textit{filename} and
compute the corresponding points in the Source Plane at redshift \textit{float}.\\
Results are put in the file "marker\_s.dat".\\
\textit{filename} must be a 3 columns ASCII file: \{ $i$ $x_i$ $y_i$ \} \\
It can be easily created with PICT using the '(g) get\_line' command.\\
The output file "marker\_s.dat" has the same format:
\{ $i$ $x^S_i$ $y^S_i$ \}\\

PICT: here is an example to visualized \textit{filename} and "marker\_s.dat":\\
{\bf
\begin{tabular}{lllll}
curve&         & & & \\
        &nfile   &2& & \\
        &namein  &1&0&\textit{filename}\\
        &column  &1&3& \\
        &        &2&3& \\
        &namein  &2&0&marker\_s.dat\\
        &column  &2&3& \\
        &        &2&3& \\
        &end     & & & \\
fini&     & & & \\
\end{tabular}
}\\


\item{{\bf radialprop \ } \textit{int} {\sl float1} {\sl float2} }\\
\textit{int \ } 0: if false 1,2,3: if true.\\
If true, will compute for the current potential some properties
of images  for a Source Plane at redshift \textit{float1}.
These properties differs according to
the value of \textit{int}. They are computed along a radial line starting at the
center of the first clump and with the position angle \textit{float2} expressed
in degree.\\
Results are put in the files: "radial.dat" and "radial2.dat". 
These are huge datafile, hence be careful!\\
If \textit{int}= 1 only "radial.dat" will be created with the following format:\\
\{ $r_i$ \ $\theta$ \ $a_i$ \ $b_i$ \ $\theta_{shear_i}$ \ $\mu_i$\}\\
where $r_i$ is the radial distance, $\theta$ is the direction of the radial
axis ($90^{\circ}$ relative to PA), $a_i$ is the radial eigenvalue of the
 magnification matrix,
$b_i$ is the orthoradial eigenvalue of the magnification matrix,
$\theta_{shear_i}$ is the direction of the shear ($90^{\circ}$ relative to PA),
  $\mu_i$ is the amplification.\\
If \textit{int}= 2 "radial.dat" and "radial2.dat" will be
created, "radial.dat"  will have  the following format:\\
\{ $r_i$ \ $\varepsilon_i$ \ $\delta_i$ \ $\tau_i$ \ $\theta_{shear_i}$ \ $\mu_i$\}\\
where $r_i$ is the radial distance, $\varepsilon_i$ is the induced
ellipticity, $\delta_i$ is the induced distortion, $\tau_i$ is the
induced deformation, $\theta_{shear_i}$ is the direction of
the shear ($90^{\circ}$ relative to PA),  $\mu_i$ is the amplification.\\
The file "radial2.dat" will have  the following format:\\
\{ $r_i$ \ $(\tau_i^2/r_i)$ \ $(\tau_i/r_i)$\}\\
If \textit{int}= 3 only "radial.dat" will be created with the following format:\\
\{ $r_i$ \ $\alpha_i$ \ $(r_i-cr)$ \ $(r_i-ct)$ \}\\
where $r_i$ is the radial distance, $\alpha_i$ is the deflection angle,
$cr$ is the radial critical radius, $ct$ the tangential critical radius.\\

\item{{\bf verbose \ } \textit{int} }\\
If \textit{int} is 0, minimal log information is printed to the screen;\\
if \textit{int} is 1, then some debugging information is printed to the screen.\\

\end{description}

\subsection{grille}

\textit{defines some parameters such has the number of potential
mode, the total
number of potential mode that are going to be test, the grid mode, and
the number of rows and columns in the grid. } \\


\begin{description}

\item{{\bf nombre \ } \textit{int} }\\
\textit{int \ } represents the number of points of the grid used to invert the
lens equation (from Source Plane to Image Plane). Must be an odd number,
typically 20 or 30. Increasing it will increase the precision in finding
all the images, but increase the computation time too. Values larger than 80
are not recommended. \\ 
Default: \textit{int}= 30.\\

\item{{\bf polaire \ } \textit{int} }\\
Set the grid to a polar shape if \textit{int}= 1, else it takes a rectangular shape.
Polar shape is advised if the main clump is centered on (0,0).\\
Default: \textit{int}=0 meaning that the program will used a cartesian grid.\\

\item{{\bf nlentille \ } \textit{int} }\\
Set the number of clumps that defines the Lens Potential. The number
of first identifier {\bf potential} must be equal or larger than this
number. If the effectively read number of potentials is lower than
\textbf{nlentille} then \textbf{nlentille} is set to the effectively
read number of potentials. \\
Default: \textit{int}=0.\\

\item{{\bf nlens\_opt \ } \textit{int} }\\ Set the number of clumps
that will be optimized in the {\bf inverse} mode.
The number of first identifiers {\bf potential} and {\bf limit}
must be equal or larger than this number otherwise \textbf{nlens\_opt} is
set to the number of \textbf{limit} identifier read.\\
Moreover one should have {\bf nlens\_opt} $\leq$ {\bf nlentille}.\\
Default: \textit{int}=0.\\

\item{{\bf nlens\_critic \ } \textit{int} }\\ In the \textsc{snake}
method to draw the critical lines, set the number of 
clumps that must be contoured by the algorithm in their order of readding in the
\verb+.par+ file. \\

\end{description}


\subsection{potential}
\label{sec.potential}
\textit{under this identifier is defined one mode of the
gravitational potential. One can define a global potential with many
modes,
for each mode a first identifier "potential" must be defined.}\\

It is allowed to put some comments after the identifier {\bf potential}, for
example:\\
{\bf potential \ \ Clump cD}\\
or\\
{\bf potential \ \ \#1}\\
This will clarify the inputfile.\\

\begin{description}

\item{{\bf  profil \ } \textit{int} }\\
Set the type of profile used to describe a clump.\\
0: circular singular isothermal sphere.\\
$$ \varphi(r)= 4\pi{\sigma_0^2\over c^2}{D_{LS} \over D_S}.r$$

1: elliptical singular isothermal sphere.\\
$$ \varphi(x,y)= 4\pi{\sigma_0^2\over c^2}{D_{LS} \over D_S}
	\sqrt{ (1-\varepsilon)x^2 + (1+\varepsilon)y^2} $$
$$ \varphi(r,\theta)= 4\pi{\sigma_0^2\over c^2}{D_{LS} \over D_S}
	r.\sqrt{ 1-\varepsilon\cos(2(\theta-\theta_0))} $$

2: circular sphere with a core radius. With profile slope {\bf exponent}.\\ 
$$ \varphi(r,\theta)= 6\pi{\sigma_0^2\over c^2}{D_{LS} \over D_S}
	r_0 \left[ 1 + (r/r_0)^2 \right]^\alpha $$
Note: If $\alpha\neq 1/2$, $\sigma_0$ does not correspond exactly to the
true 3D velocity dispersion (see eq. 3.68 of my Ph.D).\\

3: elliptical sphere with a core radius. With profile slope {\bf exponent}.\\
$$ \varphi(r,\theta)= 6\pi{\sigma_0^2\over c^2}{D_{LS} \over D_S}
	r_0 \left( 1 + (r/r_0)^2
		\left[ 1-\varepsilon\cos(2(\theta-\theta_0))\right]
		 \right)^{\alpha} $$
Note: If $\alpha\neq 1/2$, $\sigma_0$ does not correspond exactly to the
true 3D velocity dispersion (see eq. 3.68 of my Ph.D). \\
$\alpha$ must be greater than 0. \\

4: elliptical isothermal sphere with a core radius.(cf my Ph.D) \\
$$ \varphi_0= 6\pi{\sigma_0^2\over c^2}{D_{LS} \over D_S} r_0$$
$$ \varphi(r,\theta)= \varphi_0
	 \left[ \sqrt{ 1 + (r/r_0)^2} - 
	{\varepsilon}{(r/r_0)^2\over \sqrt{ 1 + (r/r_0)^2}}
		\cos(2(\theta-\theta_0))   \right]$$

5: Hubble profile... with BUGS. Do not Use!\\

6: pseudo-elliptical with core-radius and with profile slope
for the circular and elliptical part\\
$$ \varphi_0= 6\pi{\sigma_0^2\over c^2}{D_{LS} \over D_S} r_0$$
$$ \varphi(r,\theta)= \varphi_0
	\left( \left[ 1 + (r/r_0)^2 \right]^{\alpha} + 
	\varepsilon{(r/r_0)^2\over \left( 1 + (r/r_0)^2\right)^{\beta}}
	\cos(2(\theta-\theta_0))   \right)$$

Note: If $\alpha\neq 1/2$, $\sigma_0$ does not correspond exactly to the
true 3D velocity dispersion (see eq. 3.68 of my Ph.D).\\

7: Point mass.\\
$$ \varphi(r)= {4GM_0\over c^2}{D_{LS} \over D_L D_S} \log{r} $$

8: PIEMD (See Kassiola and Kovner 1993, ApJ, 417, 450)\\
The analytic potential is given by:\\
$$
{\partial^2 \Phi \over \partial x^2}={\rm Re}{\partial I^* \over
\partial x} , \ \ \ \ \ \ \ {\partial^2 \Phi \over \partial y^2}={\rm
Im}{\partial I^* \over \partial y}\ , \ \ \ \ \ \ \ {\partial^2 \Phi
\over \partial x \partial y}={\rm Im} {\partial I^* \over \partial
x}={\rm Re}{\partial I^* \over\partial y} \ ,
$$
with \\
$$
I^*=
{(1-e^2) E_0 \over 2i \sqrt{e}} 
\  {\rm ln}
\Bigl\{
{ 
{1-e \over 1+e} x -
i{1+e \over 1-e}y +
2i \sqrt{e }
\sqrt{r_0^2 +{x^2 \over (1+e)^2} + {y^2 \over (1-e)^2}}
\over (x-iy +2i r_0 \sqrt{e})
} 
\Bigr\} \ ,
$$ 
where $e=(a-b)/(a+b)$. \\
In the case of a PIEMD, the ellipticity you give is:\\
$$
\varepsilon=
3\varepsilon_\Sigma =(a^2-b^2)/(a^2+b^2)\ .
$$
The {\bf v\_disp} parameter you give is not $E_0$. But $E_0$ is computed
form {\bf v\_disp} by this way:\\
$$
E_0 = 4 \pi {D_{LS} \over D_{S}} {\sigma^2_0 \over c^2} =
6 \pi {D_{LS} \over D_{S}} {{\bf v\_disp}^2\over c^2} \ .
$$

9: Plane mass.\\

$$ \varphi(r)= {\Sigma_0\over \Sigma_{crit}}{ r^2\over 2.} $$

12: Navarro, Frenk \& White profile.

If we write the 3D mass density 

$$ \rho(r)=\frac{\rho_c}{\frac{r}{r_0}(1+\frac{r}{r_0})^2} $$

we get the lens potential

$$\varphi(r)=\varphi_0\times\left\lbrace
		\begin{array}{l}
		\ln^2(\frac{r}{2r_0})+\arccos^2(\frac{r_0}{r})
\qquad \mathrm{if}\; r \ge r_0\\
		\ln^2(\frac{r}{2r_0})-\mathrm{argch}^2(\frac{r_0}{r})
\qquad\ \mathrm{if}\; r < r_0
		\end{array}
	      \right.
$$

with

$$ \varphi_0=6\pi\frac{D_{LS}}{D_S}\frac{\sigma_0^2}{c^2}\frac{r_0}{2}$$

where we defined

$$ \sigma_0^2=\frac{8}{3}G\rho_c r_0^2 $$

which is actually not the central velocity dispersion but a caracteristic one.

The elliticity is introduced in the lens potential replacing $r$ by $r\sqrt{1-\epsilon_\varphi \cos(2(\theta-\theta_0))}$ where $\epsilon_\varphi$ is the ellipticity of the potential. The potential ellipticity is proportional to the surface density ellipticity in the small ellipticities approximation by $\epsilon_\varphi \simeq \epsilon_\Sigma / 3$ (cf. Golse \& Kneib 2002).

\vspace{1cm}

\item{{\bf  x\_centre \ } \textit{float} }\\
Set the x position of the center $x_c$. In arcseconds.\\

\item{{\bf  x\_centre\_wcs \ } \textit{float} }\\
Same as \textbf{x\_centre} but gives the position is degree WCS. This keyword
needs the presence of the \textit{reference} keyword in the \textbf{runmode} Section. \\

\item{{\bf  y\_centre \ } \textit{float} }\\
Set the y position of the center $y_c$. In arcseconds.\\

\item{{\bf  y\_centre\_wcs \ } textit{float} }\\
Same as \textbf{x\_centre\_wcs} but for the u position. \\

\item{{\bf  masse \ } \textit{float} }\\
Set the point mass $M_0$ expressed in $10^{12}$ solar masses,
only if {\bf  profil}=7.\\

\item{{\bf  pmass \ } \textit{float} }\\
Set the  mass per surface unit  $\Sigma_0$ expressed in $g.cm^{-2}$,
only if {\bf  profil}=9.\\


\item{{\bf  ellipticite \ } \textit{float} }\\
Set the ellipticity
$\epsilon_\Sigma={a^2_\Sigma-b^2_\Sigma \over a^2_\Sigma+b^2_\Sigma}$
of the mass distribution. In the program $\epsilon_\Sigma$ is converted to
$\epsilon_\varphi$ assuming that 
$\epsilon_\Sigma= 3 \epsilon_\varphi$.\\

\item{{\bf  ellip\_pot \ } \textit{float} }\\
Set the ellipticity of the potential distribution $\epsilon_\varphi= 
{a^2_\varphi-b^2_\varphi \over a^2_\varphi+b^2_\varphi}$.\\


\item{{\bf  angle\_pos \ } \textit{float} }\\
Set the position angle of the potential distribution $\theta_0$ expressed
in degree ($90^{\circ}$ relative to PA).\\
It corresponds to the direction of the semi-major axis of the iso-potential
counted from the horizontal axis, counterclockwise.\\

\item{{\bf  core\_radius \ } \textit{float} }\\
Set the core radius $r_0$, expressed in arcseconds.\\

\item{{\bf  cut\_radius \ } \textit{float} }\\
Set the cut radius $r_c$, expressed in arcseconds.\\

\item{{\bf  v\_disp \ } \textit{float} }\\
Set the central velocity dispersion $\sigma_0$ of the 3D velocity field
(supposed isotropic). Expressed in km\/s.\\
The relation between {\bf v\_disp } and the observed line-of-sight
velocity dispersion depends on the mass profile (see Wu 1993, ApJ, 411,
413) and the anisotropy factor. For circular or nearly circular
isotropic models with isothermal profile, 
$$
\sigma_{los}(0)=\sqrt{{9 \over 8}} \sigma_0=
\sqrt{{3 \over 4}}\sigma_{1D} \ .
$$
The observed line-of-sight velocity dispersion is generally obtained
from  the central galaxies of the cluster and is more or less
$\sigma_{los}(0)$. The correcting factor is therefore negligeable.
However, for other profile (non isothermal),
you have to compute by yourself the
correction. The correction is calculated for approximate King profile
in Kneib 1993 eq. 3.63\\
{\bf Caution:} in case of potential PIEMD, the parameter {\bf v\_disp}
is {\bf not} the true velocity dispersion (see PIEMD).\\


\item{{\bf  exponent \ } \textit{float} }\\
Set the exponent of the slope $\alpha$
of the potential distribution. Isothermal=0.5.\\
To use with {\bf profil}= 2,3,6.\\

\item{{\bf  alpha \ } \textit{float} }\\
Same as exponent.\\

\item{{\bf  beta \ } \textit{float} }\\
Set the exponent of the slope $\beta$
of the elliptical part of the potential distribution. Isothermal=0.5.\\
To use with {\bf profil}=6.\\

\item{{\bf  z\_lens \ } \textit{float} }\\
Set the redshift of the clumps. At present, all the clump must be at the
same redshift.\\

\end{description}

\subsection{limit}

\textit{under this identifier are defined the constraints on the p
otential
(more precisely on one mode). This identifier has to follow the
identifier
of the corresponding "potential" mode. (used only in the invert
runmode)}\\

It is advised to put the {\bf limit} identifier just after the {\bf potentiel}
identifier (for clarity).\\
As the {\bf potential} identifier, it is allowed to put comments after
the identifier {\bf limit} on the same line.\\

\begin{description}


\item{{\bf x\_centre \ } \textit{int} \textit{float1} \textit{float2} \textit{float3} }\\
Gives limits for the {\bf x\_centre} parameters 
of the deflecting potential, when using the {\bf inverse}
mode.\\
\textit{float1} is the minimum.\\
\textit{float2} is the maximum.\\
\textit{float3} is the precision desired on the parameters, be careful it is
not a dispersion!\\
\textit{int \ } tells if and how should the optimizer handle the parameters.\\
0: the optimizer do not change the parameter, and keep the value defined
within the {\bf potential} list.\\
1: consider \textit{float1} and {\sl float2} as strict bounds.\\
2: consider \textit{float1} and {\sl float2} as soft bounds. If the optimizer
find a minimum outside this bounds, he will test it.\\
3: consider \textit{float1} as a soft bound, {\sl float2} as strict bound.\\
4: consider \textit{float2} as a soft bound, {\sl float1} as strict bound.\\
-n: the optimizer will take n different values between \textit{float1} and
\textit{float2}, and try to optimize the others parameters. Only 2 parameters
can have such limit mode.\\


\item{{\bf y\_centre \ } \textit{int} \textit{float1} \textit{float2} \textit{float3} }\\
Same thing.\\

\item{{\bf ellipticite \ } \textit{int} \textit{float1} \textit{float2} \textit{float3} }\\
Same thing.\\

\item{{\bf angle\_pos \ } \textit{int} \textit{float1} \textit{float2} \textit{float3} }\\
Same thing.\\

\item{{\bf core\_radius \ } \textit{int} \textit{float1} \textit{float2} \textit{float3} }\\
Same thing.\\

\item{{\bf cut\_radius \ } \textit{int} \textit{float1} \textit{float2} \textit{float3} }\\
Same thing.\\

\item{{\bf v\_disp \ } \textit{int} \textit{float1} \textit{float2} \textit{float3} }\\
Same thing.\\

\item{{\bf exponent \ } \textit{int} \textit{float1} \textit{float2} \textit{float3} }\\
Same thing.\\

\item{{\bf alpha \ } \textit{int} \textit{float1} \textit{float2} \textit{float3} }\\
Used in models 3,6,12,84,87,88,89.\\
Same as exponent.\\


\item{{\bf beta \ } \textit{int} \textit{float1} \textit{float2} \textit{float3} }\\
Used in models 6 and 89.\\
Same thing.\\

\item{{\bf psi0 \ } \textit{int} \textit{float1} \textit{float2} \textit{float3} }\\
Same thing.\textbf{Not implemented}\\
{\bf psi0 \ } is defined for distribution mass model 0,1 by:\\
$$
{\bf psi0 } = 4\pi{{\bf v\_disp}^2\over c^2}
$$
for models 2,3,4,6,8,12:
$$
{\bf psi0 } = 6\pi{{\bf v\_disp}^2\over c^2} {\bf core\_radius }
$$
for model 7:
$$
{\bf psi0} = {4GM \over c^2 D_{OL}}\ .
$$
for model 9:
$$
{\bf psi0} = {4\pi G {\bf pmass} D_{OL} \over c^2 }\ .
$$

\item{{\bf b0 \ } \textit{int} {\sl float1} {\sl float2} {\sl float3} }\\
Same thing.\\
{\bf b0 \ } is defined for distribution mass model 0,1 by:\\
$$
{\bf b0  } = 4\pi{{\bf v\_disp \ }^2\over c^2}
$$
for model 2,3,4,6,8 by:\\
$$
{\bf b0  } = 6\pi{{\bf v\_disp \ }^2\over c^2}
$$
for model 7,9: not defined.\\

\bigskip

Note 1: if you want optimize the point mass model, the parameters you can
optimize are {\bf x\_centre y\_centre} and {\bf psi0}. {\bf psi0} is linked
to the central mass via the equation:\\
$$
{\bf psi0} = {4GM \over c^2 D_{OL}} \ .
$$
For $M$= 10 (in units of $10^{12}$ solar masses) at $z_L=0.3$
($H0=50$, $\Omega_0= 1$, $\lambda=0$) we have {\bf psi0}= 35.9\\

Note 2: If you want optimize the Plane mass model, the parameters you can
optimize are {\bf x\_centre y\_centre} and {\bf psi0}. {\bf psi0} is linked
to the central mass via the equation:\\
$$
{\bf psi0} = {4\pi G {\bf pmass} D_{OL} \over c^2 }\ .
$$
For $\Sigma_0$=.1 $g.cm^{-2}$ at $z_L=0.3$
($H0=50$, $\Omega_0= 1$, $\lambda=0$) we have {\bf psi0}=0.163\\

Note 3: It is strongly recommended to use {\bf v\_disp}
in general (expect for the point mass profile: 7).\\


\end{description}

\subsection{potfile}
\textit{under this identifier are defined the 
default parameters for all the galaxy scale mass components that account
for perturbations to the cluster potential by the galaxies. By default, the mass distribution model for the galaxies is PIEMD.}\\

\begin{description}
\item{\textbf{filein} \textit{int} \textit{filename} }

If \textit{int}=2, the galaxies catalog. must be in the following format : \\*
\begin{table}[!h]
\begin{tabular}{ccccccccc}
\textit{int} & \textit{float1} & \textit{float2} & \textit{float3} & \textit{float4} & \textit{float5} & \textit{float6} & \textit{float7} & \textit{float8} \\
$ftype$ & $x_c$ & $y_c$ & $\varepsilon$ & $\theta$ & $r_{core} \rm (kpc)$ & $r_{cut} \rm (kpc)$ & $\sigma \rm (km/s)$ & $z$\\
\end{tabular}
\end{table}

If \textit{int}=1 or 3 the format must be :\\*
\begin{table}[!h]
\begin{tabular}{ccccccccc}
\textit{string} & \textit{float1} & \textit{float2} & \textit{float3} & \textit{float4} & \textit{float5} & \textit{float6} & \textit{float7} \\
$Id: inputfile.tex,v 1.11 2008-03-04 14:20:07 ejullo Exp $ & $x_c$ & $y_c$ & $a$ & $b$ & $\theta$ & $mag$ & $Lum$\\
\end{tabular}
\end{table}

If \textit{int}=3, $x_c$ and $y_c$ are given in degrees in the World Coordinate System. \\
If \textit{int}=1, $x_c$ and $y_c$ are given in arcseconds relative to the reference point given in the \textbf{runtime} section.\\

The ellipticity ($\varepsilon$) parameter is linked to $a$ and $b$ by :
$$
\varepsilon = (a^2-b^2)/(a^2+b^2)\ .
$$
The ellipticity ($\varepsilon$) of the galaxies is then computed again according to their potential type \textbf(ftype) (cf. \ref{sec.potential}).\\

\item{{\bf type \ } \textit{int} }\\
All the potfile galaxies have the same mass profile set by \textit{int}. (See \textbf{potential} section). Default value : $81$, PIEMD.
 
In the current version, the $Lum$ value is not used.\\
 
The dynamical parameters ($r_{core}$,$r_{cut}$,$\sigma$) of the potfile galaxies are scaled from the Faber-Jackson and Tully-Fisher scaling relations for elliptical and spiral galaxies, respectively. These scaling laws conserve the mass-to-light radio of the galaxies. The scaling factors are defined below. 

\item{{\bf mag0 \ } \textit{float} }\\
\textit{float} is $m^\star$ in the scaling relations below. It can be in absolute or in relative magnitudes according to the magnitude you give in your potfile. $m^\star$ default value is $17\ \rm mag$.

\item{{\bf zlens \ } \textit{float} }\\
All the potfile galaxies with no specified redshift (Catalog format $3$) have the same redshift \textit{float}. This is used to compute the $D_{OL}$ diameter angular distance.

\item{{\bf sigma \ } \textit{int} \textit{float1} \textit{float2}}\\
\textit{float1} is $\sigma_0^\star$ in km/s. The velocity dispersion of the galaxies is given by :
$$
\sigma_0 = \sigma_0^\star \ 10^{0.4 \frac{m_\star - mag}{\sigma_{slope} } }
$$
In the \textbf{inverse} 2 optimisation method, \textit{int} set the number of bins for the potfile optimisation in the range (min,max) = (\textit{float1}, \textit{float2}). (see \textbf{inverse} section). \\
In the \textbf{inverse} 3 bayesian optimisation method, \textit{int} can be 1 or 3 for the uniform or Gaussian prior. For the uniform prior, (\textit{float1}, \textit{float2}) are the (min, max) limits. For the Gaussian prior, (\textit{float1}, \textit{float2}) are the (mean, stddev) parameters of the Gaussian pdf. \\
$\sigma_0^\star$ default value : $200\ \rm km/s$.

\item{{\bf core \ } \textit{float}}\\
\textit{float} is $r_{core}^\star$ in arc seconds. It is used to compute the core radius of the galaxies.
$$
r_{core} = r_{core}^\star \ 10^{0.4 (m_\star - mag ) 1/2}
$$

\item{{\bf corekpc \ } \textit{float}}\\
\textit{float} is $r_{core}^\star$ in kpc. It is used to compute the core radius in kpc of the galaxie. The cosmological parameters defined in the \textit{cosmology} Section are used to convert from kpc to arc seconds.
$$
r_{core} {\rm ('') \ } = \frac{1}{D_{OL}} \frac{c}{H_0} r_{core} \rm(kpc) 
$$

\item{{\bf cut \ } \textit{int} \textit{float1} \textit{float2} }\\
If \textit{int} is true, \textit{float1} is $r_{cut}^\star$ in arc seconds. The cut radius in arc seconds of a galaxy is :
$$
r_{cut} = r_{cut}^\star 10^{ 0.4 \frac{m_\star - mag}{2 \ slope} }
$$
\textit{int} and \textit{float2} are used in the potfile optimisation. (see \textbf{sigma} keyword and \textit{inverse} section). \\

\item{{\bf cutkpc \ } \textit{int} \textit{float1} \textit{float2}}\\
If \textit{int} is true, \textit{float1} is $r_{cut}^\star$ in kpc and is used to compute the cut radius of the galaxies in kpc. The cosmological parameters defined in the \textit{cosmology} Section are then used to convert from kpc to arc seconds.
$$
r_{cut} {\rm ('') \ } = \frac{1}{D_{OL}} \frac{c}{H_0} r_{cut} \rm(kpc) 
$$
\textit{int} and \textit{float2} are used in the potfile optimisation. (see \textit{sigma} keyword and \textit{inverse} section). \\

\item{{\bf slope \ } \textit{int} \textit{float1} \textit{float2}}\\
\textit{float1} is the slope value used in the $r_{cut}$ computation. \\
\textit{int} and \textit{float2} are used in the potfile optimisation. (see \textit{sigma} keyword and \textit{inverse} section). [Not yet implemented for the bayesian optimisation]. \\
$slope$ default value is 4. \\

\item{{\bf vdslope \ } \textit{int} \textit{float1} \textit{float2}}\\
\textit{float1} is the velocity dispersion slope value used in the $\sigma_0$ computation. \\
\textit{int} and \textit{float2} are used in the potfile optimisation. (see \textit{sigma} keyword and \textit{inverse} section). [Not yet implemented for the bayesian optimisation]. \\
$\sigma_{slope}$ default value is 4. \\

\end{description}
\subsection{cline}

\textit{under this identifier are defined the
parameters to compute the critical and the caustic lines.}\\

\begin{description}
\item{{\bf nplan \ } \textit{int} {\sl float} {\sl float} ...}\\
\textit{int \ } defines the number of Source Plane for which will be computed the
critical and caustic lines. The \textit{float} arguments  give the 
redshift of these planes.\\
Results are put in 2 different files: "ce.dat" (external critic and
caustic lines), "ci.dat" (internal critic and caustic lines).\\
"ce.dat" and "ci.dat" are 5-columns ASCII files with the format:\\
\{ $j$ $x^I_i$ $y^I_i$ $x^S_i$ $y^S_i$ \}.\\
With the snake algorithm, $j$ is the line identifier (we can have more than one
external or internal lines). $x^I_i$ $y^I_i$ are the coordinates of
the critical lines. $x^S_i$ $y^S_i$ are the coordinates of the corresponding
caustic lines.\\

PICT: here is an example to visualized  the external critical line ("ce.dat")
 and the internal caustic line ("ci.dat").\\
{\bf
\begin{tabular}{lllll}
curve&         & & & \\
        &nfile   &2& & \\
        &namein  &1&0&ce.dat\\
        &column  &1&5& \\
        &        &2&3& \\
        &namein  &2&0&ci.dat\\
        &column  &2&5& \\
        &        &4&5& \\
        &end     & & & \\
fini&     & & & \\
\end{tabular}
}\\

You can also use the \textit{pcl} \textsc{Perl} script.
\begin{verbatim}
pcl <clean|noclean> <ext|int> <critic|caustic>
\end{verbatim}
This script will read the "ce.dat" and "ci.dat" files and display the critical or
caustic lines on the currently  opened image in \textsc{DS9}.

\item{{\bf zonemult \ } \textit{int1} {\sl int2} {\sl filename}}\\
\textit{int1 \ } 0: if false 1: if true.\\
If true and if \textit{int} of {\bf nplan} equal $1$, will determine
for the redshift \textit{float} of {\bf nplan} the  image multiplicity
of each pixel ( \textit{int2} $\times$ {\sl int2} frame)
of the image plane (area defined by {\bf dmax \ }).
Results are written in the pixel-frame file \textit{filename}\\
If {\bf nplan} not equal $1$, {\bf zonemult \ } is not executed, and
a WARNING is displayed.\\
The format of \textit{filename} is the 'ipx' simple pixel-frame format.\\ 
NOTE : Works only with the SNAKE algorithm. \\

PICT: The following example of a PICT inputfile allow to display
the pixel-frame \textit{filename} with a gray lut starting at zgmin=0 (white)
ending at zgmax=6 (black) 
in a frame of size [x: -20. +20, y:-20. +20]\\
The position of the pixel-frame is automatically scaled (scale= 1).\\

{\bf
\begin{tabular}{lllll}
contour&         & &  \\
        &filein  &1&\textit{filename}\\
        &format  &2& \\
        &scale   &1&  \\
        &zgmin   &0&  \\
        &zgmax   &6&  \\
        &end     & & \\
frame&         & &  \\
        &dmax    &20.&  \\
        &end     & & \\
fini&     & & \\
\end{tabular}
}\\


\item{{\bf dmax \ } \textit{float}}\\
Defines the area (xmin= -\textit{float}, xmax={\sl float}; ymin=-{\sl float},
ymax=\textit{float}) where the critical lines are search.
\textit{float} is expressed in arcseconds. A typical value of {\bf dmax} is
30. \\
Default: the value defined in {\bf champ}.\\

\item{{\bf algorithm \ } \verb+marchingsquares|snake+ }\\
Select one of the two algorithms for the computation of the critical and caustic 
lines.\\
The snake algorithm is the original algorithm implemented in \lenstool.
For each of the first \textit{nlens\_crit} clumps of the lens model, the algorithm
starts from the centre of the clump and looks for a point on a surrounding critical 
line (locus of the space where amplification is infinite). Then, it tries to follow
 this critical line and to go back to its original starting point.\\

The marching squares algorithm defines a first square with the \textit{dmax} or 
the \textit{champ} keywords. Then, it divides this first cube in 4 small squares.
According to their size and their amplification values in the centre and the 4 corners, each square is divided or not in 4 further small squares.\\
If the field is rectangular, the greater value between the width and the height of the field is choosen as the square size.\\
As long as the size of a square is greater than \textit{limitHigh}, it is 
automatically divided in 4 small squares. The size of a square cannot be lower than \textit{limitLow}.\\
The marching squares algorithm is slower than the line-following snake algorithm but gives always the full contour of the critical lines. It is less sensitive to small irregularities in the contour. The snake algorithm always returns a connected contour.\\
Default algorithm is \verb+marchingsquares+.\\


\item{{\bf pas \ } \textit{float}}\\
For the line-following algorithm :\\
Defines the step between each search {\em i.e.} it represents the typical
distance in arcsecs between each point of the external critical lines.
For internal critical lines half this value is taken.\\
To improve the definition of the critic and caustic lines, use smaller
values such as 0.5" or even 0.2".\\
Default is 1."\\

\item{{\bf limitLow \ } \textit{float}}\\
For the marching cube algorithm :\\
Defines the smaller size of a square {\em i.e.} the size 
of a square is compared to this value to decide between 
dividing again in 4 squares or stopping the division.\\
To improve the definition of the critic and caustic lines, use smaller 
values such as 0.5" or even 0.2". This implies more computation time.\\
Default is 1".\\

\item{{\bf limitHigh \ } \textit{float}}\\
For the marching cube algorithm :\\
Defines the higher size of a square {\em i.e.} the size of a square cannot 
be higher than this value. A square
with a size above this value is automatically divided in 4 squares.\\
Decrease this value to remove holes in the critical lines and improve the
detection of critical lines around isolated galaxies. This implies more
computation time.\\
Default is 10".\\

\end{description}

\subsection{grande}

\textit{ under this identifier is defined the way to represent th
e
computed deformation of objects.}\\

\begin{description}


\item{{\bf large\_dist \ } \textit{float} }\\
\textit{float \ } set the value for which we can consider we have a strong
deformation (it corresponds to a minimum value of $\tau_I$). Typical value
is 1. or 2.\\

\item{{\bf profil \ } \textit{int1} {\sl int2} }\\
\textit{int1 \ } 0: if false 1: if true.\\
If true, set the representation mode of large distorted source object
to a density points where \textit{int2} is
the number of points. A Gaussian profile for the source is assumed.\\
Results is a list of points in the image plane that are stored in 
the ASCII file: \textit{gianti.dat}. The \textbf{profil} keyword has 
no effect if the \textbf{contour} keyword is true.\\

PICT : The gianti.dat file can be displayed on ds9 with the \textit{gianti}
perl script.

\item{{\bf contour \ } \textit{int1} {\sl int2} }\\
\textit{int1 \ } 0: if false n: if true.\\
If true, set the representation mode of large distorted source object
 to contour points. If non zero, \textit{int1} set the number 
 of isocontours for the source. \textit{int2} set the number 
 of points per isocontour.\\
Results is a list of points in the image plance that are stored in 
the ASCII file: \textit{gianti.dat}.\\

PICT : The gianti.dat file can be displayed on ds9 with the \textit{gianti}
perl script.

PICT: The following example of a PICT inputfile allow to display
the gianti.dat file.\\
{\bf
\begin{tabular}{lllll}
curve&         & & &  \\
        &nfile   &1 & & \\
        &namein  &1&1&gianti.dat\\
        &format  &1&1& \\
        &end     & & & \\
fini&     & & & \\
\end{tabular}
}\\
Note that the second sinteger in namein, corresponds
to the way to trace the list of points:\\
0 is a line\\
n>0 individual points (the value of n sets the type of points)\\
n<0 individual points and a line (the value of n sets the type of points)\\



\item{{\bf iso \ } \textit{int1} {\sl int2} {\sl float1} {\sl float2} {\sl float3} }\\
\textit{int1 \ } 0: if false 1,2: if true.\\
If  true set the representation of large distorted object to an Image mode.\\
If \textit{int1}= 1, set the initial window to the minimum size that include
the position of the center of images, and their computed sizes.\\
If \textit{int1}= 2, set the initial window to the one defined in {\bf champ}.\\
\textit{int2 \ } is the maximal number of pixel tolerate for the final image.\\
\textit{float1 \ } is the pixel size desired for the image.\\
\textit{float2 \ } and {\sl float3} are extension factors, from the initial window.
(a value of 0.5 will add half the total size to both size.)\\
Note: contrary to {\bf pixel}, {\bf iso} make gravitational images of
only one source. The \textbf{iso} keyword has 
no effect if any of the \textbf{contour} or the \textbf{profil} keywords are true. \\

\item{{\bf name \ } \textit{filename} }\\
Generic name for the final image, when using {\bf iso} mode. It write for
each arclets the results in the pixel-frame file \textit{filename}{\bf n},
where $n$ is the index of the arclet.\\
Default value of \textit{filename} is {\sl giant}\\

PICT: The following example of a PICT inputfile allow to display
the pixel-frame \textit{filename}{\bf n} with a gray lut starting
at zgmin=0 (white) ending at zgmax=50 (black) 
in a frame of size [x: -20. +20, y:-20. +20]\\
The position of the pixel-frame is automatically scaled (scale= 1).\\
{\bf
\begin{tabular}{lllll}
contour&         & &  \\
        &filein  &1&\textit{filename}{\bf n}\\
        &format  &2& \\
        &scale   &1&  \\
        &zgmin   &0.&  \\
        &zgmax   &50.&  \\
        &end     & & \\
frame&         & &  \\
        &dmax    &20.&  \\
        &end     & & \\
fini&     & & \\
\end{tabular}
}


\item{{\bf vitesse \ } \textit{int} }\\
\textit{int \ } 0: if false 1: if true.\\
If true, the profile of the source will not be taken as a gaussian profile
but will have a velocity profile. The intensity will be proportional
to the distance of the small axes of the ellipse.\\
The main purpose of {\bf vitesse} is to look at possible velocity
gradient along giant arcs due to internal velocity field of the lensed
galaxy.\\

\end{description}

\subsection{observ}
 
\textit{under this identifier is defined the different noise that
t can
be add to a gravitational image, such as seeing or Poisson Noise.}\\

All these constants are used when the image mode {\bf grande iso} or
{\bf runmode pixel} representation is set.\\


\begin{description}

\item{{\bf bruit \ } \textit{int}}\\
\textit{int \ } 0: if false 1: if true.\\
If true add Poisson noise to the final image.\\


\item{{\bf sky \ } \textit{float}}\\
mean value for the sky background.\\

\item{{\bf idum \ } \textit{int}}\\
\textit{int \ } random number, should be negative.\\

\item{{\bf dispersion \ } \textit{float}}\\
1 $\sigma$ value of the background.

\item{{\bf prec \ } \textit{float}}\\\
\textit{float \ } defines the precision for the calculation of the
value of each pixel, $0.1$ is a typical value.\\

\item{{\bf seeing \ } \textit{int} {\sl float}}\\
\textit{int \ } 0: if false 1: if true.\\
If true will convolve the image by a Gaussian filter with a
full width at half maximum of size \textit{float \ } expressed in arcseconds.\\
Note: for good accuracy the pixel size of the image must be small compared
to the seeing, a minimum of 2 pixels, but with 4 pixels as an optimum.


\item{{\bf binning \ } \textit{int1} {\sl int2}}\\
\textit{int \ } 0: if false 1: if true.\\
If true will "bin" the image by \textit{int2} pixels.\\
Note: If the binning is set is done just after the seeing and before to add
noise. The aim is to better calculate the seeing when the sampling of final
image is low.\\

\end{description}

\subsection{cosmologie}

\textit{under this
identifier are defined the cosmological parameters
$\Omega_0$, $\lambda$ and $H_0$.}\\

\begin{description}
\item{{\bf H0 \ } \textit{float} }\\
\textit{float \ } defines the value of H0 in Mpc/km/s.\\
Default value is $H0=50$.\\

\item{{\bf omega \ } \textit{float} }\\
\textit{float \ } defines the value of $\Omega_0$.\\
Default value is $\Omega_0= 1$.\\

\item{{\bf lambda \ } \textit{float} }\\
\textit{float \ } defines the normalized value of $\lambda$. (for a flat universe
$\Omega_0$ + $\lambda$ = 1.)\\
Default value $\lambda = 0$.\\


\end{description}

\subsection{cosmolimit}
\textit{Under this identifier are defined the limits on the cosmological parameters that you want to optimise during the Bayesian optimisation.}  \\

\begin{description}
\item{{\bf omega\ }or {\bf omegaM\ } \textit{int} \textit{float1} \textit{float2} }\\
Gives limits for the $\Omega_{M}$ parameter.\\
\textit{int} set the prior pdf. 1: Uniform, 3: Gaussian. \\
With a Uniforma prior, \textit{float1} and \textit{float2} are the lower and upper limits.\\
With a Gaussian prior, \textit{float1} and \textit{float2} are the mean and stddev parameters of the Gaussian pdf. \\

\item{{\bf omegaX\ }or {\bf lambda\ } \textit{int} \textit{float1} \textit{float2} }\\
Gives limits for the $\Lambda$ parameter.\\
\textit{int} set the prior pdf. 1: Uniform, 3: Gaussian. \\
With a Uniforma prior, \textit{float1} and \textit{float2} are the lower and upper limits.\\
With a Gaussian prior, \textit{float1} and \textit{float2} are the mean and stddev parameters of the Gaussian pdf. \\

\item{{\bf wX\ } \textit{int} \textit{float1} \textit{float2} }\\
Gives limits for the $w_X$ parameter.\\
\textit{int} set the prior pdf. 1: Uniform, 3: Gaussian. \\
With a Uniforma prior, \textit{float1} and \textit{float2} are the lower and upper limits.\\
With a Gaussian prior, \textit{float1} and \textit{float2} are the mean and stddev parameters of the Gaussian pdf. \\

\end{description}

\subsection{champ}

\textit{under this identifier is defined the size of the field use
d in some
calculations such as the dimension of the grid for the inversion of the
lens
equation.}\\

\begin{description}
\item{{\bf xmin \ } \textit{float} }\\
\textit{float \ } give the minimal value in X (in arcseconds) of the frame where computation are
made.\\

\item{{\bf xmax \ } \textit{float} }\\
\textit{float \ } give the maximal value in X (in arcseconds) of the frame where computation are
made.\\

\item{{\bf ymin \ } \textit{float} }\\
\textit{float \ } give the minimal value in Y (in arcseconds) of the frame where computation are
made.\\

\item{{\bf ymax \ } \textit{float} }\\
\textit{float \ } give the maximal value in Y (in arcseconds) of the frame where computation are
made.\\

\item{{\bf dmax \ } \textit{float} }\\
Quick definition of the 4 limits (in arcseconds): {\bf xmin}=-\textit{float},
{\bf xmax}=\textit{float}, {\bf ymin}=-{\sl float} and {\bf ymax}={\sl float}.\\

\end{description}

\subsection{cleanlens}

\textit{under this identifier are defined some parameters to
retrieve the shape of the source knowing a pixel-frame of the image.}\\

Here the program reads real CCD pixel-frame and using the equation of the
lens compute a pixel-frame of the source. For each point of the
Image plane the program can compute the corresponding point in the Source Plane.
Then for each pixel with multiplicity $n_{ij}$ of the source plane
we then can attribute an intensity computed as the mean of the intensity of
corresponding image pixels:
$$
{\cal I}^S_{ij}= {1\over n_{ij}} \Sigma_{k=1}^{k=n_{ij}} {\cal I}^I_{ijk}
$$
The error of this reconstruction is given at position $ij$ in the source
plane by:\\
$$
e_{ij}= (n_{ij}-1) \Sigma_{k=1}^{k=n_{ij}} ({\cal I}^I_{ijk}- {\cal I}^S_{ij})^2
$$
(the error estimate exists only if we have multiple images).\\

When using the {\bf inverse} mode, the program will minimize the
estimate of the error ($e=\Sigma e_{ij}$). This method nevertheless
numerically different, was first described by Kochaneck \textit{et al.} 1989.\\ 


\begin{description}

\item{{\bf cleanset \ } \textit{int} {\sl float} }\\
\textit{int \ } 0: if false 1: if true.\\
If true the program will compute from the image pixel-frame
{\bf imframe \ } the corresponding
source frame {\bf sframe \ } at redshift \textit{float}.\\
If the inverse mode is selected see {\bf runmode inverse}
 an optimization of the 'ring cycle' form will be done.
In this case
the optimization will work only if multiples images or giant arcs
are present in the image pixel-frame. It is strongly advise the user to have an
a priori good guess on the mass model parameters.\\
Moreover it will create the pixel-frame:\\
- \textit{erreur.ipx} : this frame is relevant only in the case of giant
or multiple arcs, it compute the reconstruction error of the source pixel-frame
for each pixels.
- \textit{imult.ipx} : this pixel-frame show the multiplicity of each pixel of the
reconstructed source pixel-frame.\\
\item{{\bf c\_image \ } \textit{filename} }\\
If \textit{int} = 1 for the \textbf{cleanset} keyword, \textit{filename} is an ASCII file that contains a list of points {$i,$, $x_i$, $y_i$} in the image plane given in arcsec relative to the
image reference points (see \textit{reference} keyword in \textit{runmode} section)
 that delimit the
center of the observed image. The barycenter of those points is sent to the source
plane and defines the source center. This value is used to compute the WCS keywords
of the resulting source FITS file.\\

\item{{\bf imframe \ } \textit{int} \textit{filename}}\\
\textit{int} format of the input image ( {\sl int}= 2 for ipx format,
3 for fits file).\\
\textit{filename} name of the CCD frame where (multiple)
gravitational images are present.\\

\item{{\bf psfframe \ } \textit{int} {\sl filename}}\\
\textit{int} format of the input Point Spread Function (PSF)( {\sl int}= 2 for
ipx format, 3 for fits file).\\
\textit{filename} name of the PSF frame, namely a Star Profile.
The Star must be at the center of the frame. And the total intensity
of the frame (sum of all the pixels) must be equal to 1 (PSF normalized).\\ 
Note: the psf frame is used when using a deconvolve-inversion of the
lensed images. (not working yet)\\

\item{{\bf sframe \ } \textit{filename}}\\
\textit{filename} name of the output source frame.\\
This frame will correspond to the inversion of the {\bf imframe \ }
input frame (from Image Plane to Source Plane).\\
It is written with the ipx format.\\

\item{{\bf ncont \ } \textit{int} {\sl filename}}\\
\textit{int} number of contour.\\
\textit{filename} name of the output frame: that is the CCD frame where
only the pixel inside the contours (closed lines) have been kept.\\
The idea is to limit the area of the frame, and keep only the interesting
pixels.\\

\item{{\bf contour \ } \textit{int} {\sl filename}}\\
Define contours in the image plane of one or several images of the source you want the
shape in the source plane.
\textit{int} index of the contour for one image. First contour must be index by 1, second by 2, etc.\\
\textit{filename} is the name of the contour ASCII  file for one image that contains a list of points {$i,$, $x_i$, $y_i$} in the image plane given in arcsec relative to the
image reference points (see \textit{reference} keyword in \textit{runmode} section).

\item{{\bf echant \ } \textit{int} }\\
It is possible to subsample the CCD frame to calculate with greater
accuracy the source frame in the source plane. \textit{int} is the 
subsampling parameter. Default is 1. If \textit{int}= 2 it will 
cut each pixel in 4 smaller pixel. Default value : 2.\\

\item{{\bf s\_echant \ } \textit{int} }\\
It is possible to subsample the frame in the source plane as you can do 
in the image plane with the \textit{echant} keyword. If \textit{int} is too 
high, you can get an image with a undefined pixel values. Default value : 1.\\

\item{{\bf s\_n \ } \textit{int} }\\
Define the width and height of the resulting source image in pixels. Default 
value : 50 pixels.\\

\end{description}

The following identifier should be set if the format of the 
{\bf imframe} is 0 (ascii format) or 1 (ipx format without scaling).
It is strongly advised to used
the ipx format: 2 or fits format: 3, to avoid to define such identifier.\\

\begin{description}

\item{{\bf pixel \ } \textit{float} }\\
set size of the pixel in x and y in arcseconds.\\

\item{{\bf column \ } \textit{int} }\\
column to be selected (in a multi-column ascii pixel-frame file).\\

\item{{\bf header \ } \textit{int} }\\
number of line to skip before the real beginning of the data.\\

\item{{\bf pixelx \ } \textit{float} }\\
pixel size in x in arcseconds (if the pixel is not a square).\\

\item{{\bf pixely \ } \textit{float} }\\
pixel size in y in arcseconds (if the pixel is not a square).\\

\item{{\bf xmin \ } \textit{float} }\\
x position of the bottom-left pixel (expressed in pixel units).\\

\item{{\bf ymin \ } \textit{float} }\\
y position of the bottom-left pixel (expressed in pixel units).\\


\end{description}

\subsection{image}

\textit{ under this identifier are defined some characteristics of
 images,
multiple images or arclets.}\\

\begin{description}

\item{{\bf multfile \ } \textit{int} {\sl filename}}\\
\textit{int \ } 0: if false, 1: if true.\\
If true will used the multiple images defined in \textit{filename} to optimize
the lens parameters and/or the unknown redshift(s) of multiple images.\\
The format of \textit{filename} is an 'ellipse'-like format:\\
\{ $i$ $x_{ij}$ $y_{ij}$ $a_{ij}$ $b_{ij}$ $\theta_{ij}$ $z_i$ $mag$\}\\
where $i$ is an identifier for all the multiple images 
of a single source at redshift $z_i$. (see chapter datafile). $\theta_{ij}$ is
 $90^{\circ}$ relative to PA. $mag$ is the magnitude of the image. It is used 
in the flux optimisation mode.\\
In the same file the user can put different families of multiple images.
Here is an example of 2 families of multiple images:\\
{
\bf
C1a 20.5 30.4 1.2 0.6 40. 0.65 0.\\
C1b 30.8 20.3 1.3 0.6 50. 0.65 0.\\
C1c 25.1 25.2 1.2 0.5 60. 0.65 0.\\
C2a 10.3 -5.2 1.2 0.9 10. 0. 0.\\
C2b -9.7 -4.2 1.4 0.5 20. 0. 0.\\
}\\
The redshift of the second family of multiple images in this example
is not known, therefore the redshift is set to 0 and its convergence
can be constraint by the {\bf z\_m\_limit} identifier (see below).\\

\item{{\bf mult\_wcs \ } \textit{int} }\\
If true, the multiple images coordinates are considered absolute WCS coordinates. They are transformed to relative coordinates with the \textit{reference} keyword position given in the \textbf{runmode} section.\\
If false, their coordinates are considered relative in arcsec.\\

\item{{\bf forme \ } \textit{int} }\\
\textit{int} 0: if false, 1: if true.\\
If true will used both the position and the ellipticities of
the \textbf{multfile} as constraints. If false will only used
the position of the \textbf{multfile} as constraints. In both cases, optimisation
is done in the source plane.\\
Default value is -1 for the optimisation in the image plane.\\


\item{{\bf z\_m\_limit \ } \textit{int1} \textit{imageId} \textit{int3}
\textit{float1} \textit{float2} \textit{float3} }\\
\textit{int1} 0: if false 1: if true.\\
If true will optimize the redshift of the multiple images of index
\textit{imageId} corresponds any of image identifiers of a given source given in the \textbf{multfile} filename (ex: C2b or A1).\\

\textit{int3} determines the properties of the boundaries \\
In parabolic optimisation, its meaning is :  1:strict, 2:soft, 3: left soft right strict, 4: right soft left strict, -n: sampling.\\
In Bayesian optimisation, its meaning is : 1: uniform prior, 3: Gaussian prior.\\

\textit{float1} lowest boundary or mean value (Gaussian prior).

\textit{float2} highest boundary or sttdev value (Gaussian prior).

\textit{float3} precision to reach to stop the optimization of the redshift. (Not considered in Bayesian optimisation).

This identifiers is very similar to the second identifiers of {\bf limit}.\\

\item{{\bf arcletstat \ } \textit{int1} {\sl int2} {\sl filename}}\\
\textit{int1 \ } 0: if false, 1: if true.\\
If true will used the arclets in \textit{filename} to optimize the lens
parameters assuming all the sources at the redshift defined
in {\bf source} first identifier (See sect. 2.2.12).
The format of the catalogue supplied in \textit{filename} depends on the value of {\it int2}. If {\it int2 } 0: it is the same as the
arclet catalogue referred to 
in section~\ref{sect:secondid:runmode} (i.e.\ an ASCII column 
format of the form:
\{ $i$ $x_i$ $y_i$ $a_i$ $b_i$ $\theta_i$ $z_i$ \}, where
$\theta_i$ is associated with the ellipses' major axis, is in degrees and is 
measured anticlockwise from West, and $x$ and $y$ are RA
and Dec in decimal degrees). If it is {\it int2} = 2, it needs 2 extra columns { $vare1_i$ $vare2_i$ }, with the shape measurement variance on $e1$ and $e2$. Note that the redshifts of the sources are defined
in this catalogue -- it is up to the user to provide sensible estimates for 
them. \\

\textit{int1} is actually more than just a switch: it defines 
the form  of the likelihood used in the optimisation. 
The recommended mode is 6 -- which is to
use a likelihood based on the assumed distribution of intrinsic source plane
complex ellipticities. 
\\

\item{{\bf sigell \ } \textit{int1}  {\sl float} \textit{int2} }\\
If true, the width of the (assumed Gaussian) intrinsic ellipticity distribution
can be specified here (the default value is 0.3). The sum of its square enters and eventually 
a shape measurement variance constitutes the denominator of the $\chi^2$ calculation.\\

\item{{\bf z\_arclet \ } {\sl float}}\\
In the case of {\bf arcletstat \ } it is possible to fix the
redshift \textit{float}
of the arclets with unknown redshift (redshift value set to 0 in the {\bf arcletstat \ }
\textit{filename}).\\

\item{{\bf z\_a\_limit \ }�{\sl int1} {\sl float1}�{\sl float2}}\\
If {\sl int1} : 1, assumes the arclets with unknown redshift in {\bf arcletstat \ } {\it filename} 
(redshift set to 0), needs to be optimized with a flat prior with min and max boundaries
{\sl float1} and {\sl float2}. If {\sl int1} : 3, assumes a Gaussian prior with mean and width 
{\sl float1} and�{\sl float2}. {\sl int1} : 0,  if no optimization.

\item{{\bf critic \ } \textit{int1} {\sl float1} {\sl float2} {\sl float3} {\sl float4} {\sl float5} }\\
\textit{int1 \ } 0: if false 1: if true.\\
If true will add the constraint of 
the position of the break (locus of merging images)
and the orientation of the
image at the break (i.e. the direction of amplification matrix).\\
\textit{float1 \ } x position (in arcseconds) of the break.\\
\textit{float2 \ } y position (in arcseconds) of the break.\\
\textit{float3 \ }  direction of the orientation of the image at the break point,
expressed in degree from the horizontal line, counter-clockwise.\\
\textit{float4 \ } error of position (in arcseconds) of the break along the
arc.\\
\textit{float5 \ } redshift of the source of the merging images.\\

Note: it is possible to give more than one such a constraint, from merging
points at different position and different redshifts. One has simply to
enter as many {\bf critic \ } lines as merging points.\\

\end{description}

\subsection{source}

\begin{description}
\item{{\bf grid \ } \textit{int} }\\
\textit{int \ } 0: if false 1: if true.\\
If true the sources are placed on a regular grid in the source Plane.\\

\item{{\bf random \ } \textit{int} }\\
seed for the random number generator, better if negative.\\

\item{{\bf n\_source \ } \textit{int} }\\
Number of sources to draw.\\

\item{{\bf elip\_max \ } \textit{float} }\\
Maximum ellipticity of the drawing sources. The ellipticity
of the sources is drawn from 0 to \textit{float} with a uniform
law.\\

\item{{\bf dist\_z \ } \textit{int} }\\
\textit{int \ } 0: if false 1: if true.\\
If true will draw the redshift with a uniform law between {\bf z\_source \ }
and 2{\bf z\_source \ }.\\

\item{{\bf z\_source \ } \textit{float} }\\
redshift of the sources.\\

\item{{\bf taille \ } \textit{float} }\\
set the size of the source in arcsecond.\\

\end{description}

\subsection{fini}

\section{Examples}

\subsection{Typical configurations of Arcs}

\subsection{Optimization with one multiple image}

\subsection{Optimization with two multiple images}

\subsection{Optimization with arclet data}

The directory \texttt{lenstool/example\_with\_arcletstat} contains a simulated
weak lensing catalogue, \texttt{nfw\_arclet.cat}, in standard form. The model
used to create this was an elliptical NFW potential, with true parameters as
follows: $M_{200} = 2.0\times10^{15} M_{\odot}$, $c_{200} = 7.0$,
$|\epsilon_{\psi}| = 0.15$, $\theta_{\psi} = 60.0$.  The origin of the catalogue
coordinate system is at the centre of the lens potential. The source galaxy
positions are the same as those in the central 5 arcmin of the MS0451 catalogue
of Smith et al 2008 (in prep); their ellipticity components were drawn from a
Gaussian  distribution of width 0.25, and then transformed using the expressions
of Seitz \& Schneider (2002). Gaussian shape estimation noise was then added to
each ellipticity component, assuming an rms of 0.2.

This example comes with two parameter files, \texttt{nfw.par} and
\texttt{gnfw.par}, for fitting the NFW and gNFW profile models respectively to
the same shear data.
This is a situation that arises very fequently in data analysis; the script
\texttt{run.csh} was written to illustrate the recommended way of running
\lenstool, i.e.\ in a directory for each fit (where the fit is defined by the
parameter file). You can try out the two model fits by executing \\ 
   \texttt{run.csh} nfw\\
and 
   \texttt{run.csh} gnfw\\
Examples of typical output files are given in the 
\texttt{ref\_nfw} and \texttt{ref\_gnfw} directories for comparison.\\
{\bf TODO: Dave, Eric: we need to test these models to our satisfaction, and
then check in the example results into the ref directories for the users.}\\ 
 
