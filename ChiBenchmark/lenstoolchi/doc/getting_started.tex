\input macro.tex
\parindent0pt
\chapter{Getting Started}

\subsection*{Initialization}

$\bullet$ Visualize the image of the cluster and choose the origin of the
Image Plane. For example you can choose it at the barycenter of the
central galaxies light. Then scale your image to this position and give
to the pixel-size its real size in arcseconds. \\

$\bullet$ Note the positions (x,y),
the orientation and the ellipticities of all the galaxies you plan to
use for the modeling. You can add galaxies that you will not use as
deflectors, but only to recognize your field in your figures. These data
will be included in the file *.par that you generate at the beginning,
before starting the runmode. \\

$\bullet$ Note the positions, the orientations and the ellipticities of
all the gravitational images (arcs or arclets) you plan to use for the
modeling. Note carefully, those that come from a same source. In case of
giant arcs in which you can readily see the double (fold) or triple
(cusp) component, you can consider small ellipses within the arcs as
multiple images also. \\

\subsection*{Optimization}

$\bullet$ multiple images optimization.\\

$\bullet$ arclets optimization.\\

$\bullet$ break constraints: if images are good,
try to find the position where images are
merging, if any. You  must note the position, the direction of the
orientation of the image at the break (merging point), the error of
position at the break.\\

$\bullet$ cleanlens methods\\

\subsection*{PICT visualization}

The visualization of the CCD image and the parameters can be obtained
with PICT, CPICT, or SAOIMAGE for instance. These environment give the
orientation and the ellipticities of any objects in the field
interactively.\\


Once the parameters are inserted in files, you are ready to play with
ngrille.\\



