\input macro.tex
\chapter{Datafiles}

\section{Input Datafiles}

\subsection{WCS header}
In every input file, it is possible to set a WCS header to define the coordinates of the objects in the file. \\
{\sl \#REFERENCE \ } {\sl int \ RA \ DEC} \\
If \textit{int} = 0 : the positions are in degree WCS aligned.\\
If \textit{int} = 1 : the positions are in arcsec relative to (RA,DEC) expressed in sexagesimal format (HH:MM:SS DD:MM:SS). \\
If \textit{int} = 2 : the positions are in pixels relative to the reference pixels. In this case, (RA,DEC) defines the coordinates (X,Y) of the reference pixel. \\
If \textit{int} = 3 : the positions are in arcsec relative to (RA,DEC) expressed in degrees. \\

\subsection{Object file}

\subsubsection{Definition}

This ascii file contains a list of objects characterized by their position,
shape parameters and redshift, with the following format.


There are 2 formats. The default one specified by the first argument
in the parameter file {\sl int1} = 1 is  

{\sl int \ \  float1 \ \ \ float2 \ \ \ float3 \ \ \ float4
\ \ \ float5 \ \ \ float6}

{\sl int \ \ } is a characteristic integer that defines the object.

{\sl float1 \ \ } is the X position of the object expressed in arcsec or in degree.

{\sl float2 \ \ } is the Y position of the object expressed in arcsec or in degree.

{\sl float3 \ \ } is the semi-major-axis $a$ of the equivalent ellipse of the 
object, expressed in arcsecond.

{\sl float4 \ \ } is the semi-minor-axis $b$ of the equivalent ellipse of the 
object, expressed in arcsecond.

{\sl float5 \ \ } is the position-angle $\theta$ of the equivalent ellipse of the 
object, expressed in degree. This give the orientation of the semi-major-axis
from the horizontal line (counter-clockwise).

{\sl float6 \ \ } is the redshift $z$ of the object. If the redshift is unknown
this value should be set to $0$.

{\sl float7 \ \ } is the magnitude of the object.

The alternate one specified by the first argument {\sl int1} = 2 (see
below) is useful in conjunction with keyword arcletstat for weak
lensing. It is similar to the default format, with the 2 extra columns

{\sl float8 \ \ } is the variance of the E1 ellipticity component
(ellipticity defined as $E = a^2 - b^2 / a^2 + b^2$.)

{\sl float9 \ \ } is the variance of the E2 ellipticity component.

\subsubsection{Uses}

Such a file can characterize either an arclet file or a source file.
Here is the list of identifiers which requires such a file:

{\bf runmode \ \ \ \ image \ \ \ } if $z=0$ then the program will use the
one defined by {\bf source \ \ \ \ z{\_}source \ \ \ }.

{\bf runmode \ \ \ \ source \ \ \ } if $z=0$ then the program will use the
one defined by {\bf source \ \ \ \ z{\_}source \ \ \ }.

{\bf runmode \ \ \ \ study \ \ \ } the value of $z$ is not used under this
identifiers.

{\bf image \ \ \ \ multfile \ \ \ } for each image in each set of multiple
images the characteristic integer and the redshift ($0$ if unknown)
must be the same.

{\bf image \ \ \ \ arcletstat \ \ \ }

\subsection{Marker file}

This ascii file contains a list of points, with the following format:

{\sl int \ \  float1 \ \ \ float2}

{\sl int \ \ } is a characteristic integer that defines the marker.

{\sl float1 \ \ } is the X position of the marker expressed in arcsec.

{\sl float2 \ \ } is the Y position of the marker expressed in arcsec.

\subsubsection{Uses}

Such a file can characterize only markers that are in the Image Plane.
There is only one identifier that uses such a file. That is:

{\bf runmode \ \ \ \ marker \ \ \ }

\subsection{IPX pixel-image file}

The IPX format is a simple format for pixel-images. It is made of an
ASCII header of 4 lines, followed by the data, that can be written
either in ASCII or in binary.\\

The header is defined in this way:\\

{
\bf
\begin{tabular}{lllll}
2&xmin&xmax&ymin&ymax\\
nx&&&& \\
ny&&&& \\
type&mode&nature&&\\
comments&&&&\\
\end{tabular}
}\\

2 stands for Dimension 2.\\
xmin, xmax, ymin, ymax are floats value defining the scaling of the image.
The center of the bottom-left pixel is (xmin,ymin), the center of the 
upper-right pixel is (xmax,ymax).\\
nx is the dimension in x, ny in y.\\
type can be {\sl int} or {\sl float} or {\sl double}, it is the type of the
pixel-image data.\\
mode can be {\sl txt} for an ASCII representation of the data, or
{\sl bin} for a binary representation.\\
nature is either {\sl real} or {\sl complex}.\\
comments is at maximum a 1024 long ASCII chain, ended by an EOL character.
If mode is set to {\sl txt} the data is listed in the file as a column.
If it is set to {\sl bin} the data can be read line by line.


\subsection{FITS pixel-image file}

The program can read FITS pixel-frame. It will read the FITS file and
convert the data in float (whatever was the type of the data in the file).\\

The pixel-frame has to be scaled.\\

This is done in FITS by modifying the following keywords:\\
CRVAL2 = x-value of pixel CRPIX2\\
CRPIX2 = index  $i$ of the reference pixel\\
CDELT2 = pixel size in the x direction\\
CRVAL1 = y-value of pixel CRPIX1\\
CRPIX1 = index  $j$ of the reference pixel\\
CDELT1 = pixel size in the y direction\\

From this values the program compute the xmin,xmax,ymin,ymax in this way:\\

xmin = CRVAL2 + (CRPIX2-1)*CDELT2\\
xmax = CRVAL2 + (nx - CRPIX2)*CDELT2\\
ymin = CRVAL1 + (CRPIX1-1)*CDELT1\\
ymax = CRVAL1 + (ny - CRPIX1)*CDELT1\\



\section{Output Datafiles}

{\bf ngrille} can generate a lot of output file, depending on what
was given in the inputfile. It generates always a "mouchard" file
{\sl para.out \ }. Where we can find back the different identifiers with their
values, plus others computed constants.

The following subsections give the complete list of the output file that
can be created by {\bf ngrille} with their format.


\subsection{Potential file}

{\sl pot.dat \ \ \ }

This ascii file is written with the following format:

{\sl int \ \  float1 \ \ \ float2 \ \ \ float3 \ \ \ float4
\ \ \ float5 \ \ \ float6}

{\sl int \ \ } is a characteristic integer that defines the potential clump.

{\sl float1 \ \ } is the X position of the center of the clump
expressed in arcsec.

{\sl float2 \ \ } is the X position of the center of the clump
expressed in arcsec.

{\sl float3 \ \ } is the semi-major-axis $a$ of the ellipse of the 
line, expressed in arcsecond.

{\sl float4 \ \ } is the semi-minor-axis $b$ of the ellipse of the 
line, expressed in arcsecond.

{\sl float5 \ \ } is the position-angle $\theta$ of the ellipse of the 
line, expressed in degree. This give the orientation of the semi-major-axis
from the horizontal line (counter-clockwise).

{\sl float6 \ \ } is a non-significative constant, in general $0$.


The lines represented are the core{\_}radius if any with the ellipticity
and the orientation of the mass distribution, and the tangential
critical line of the clump if it was alone (analytic expression).
Furthermore a cross indicates the center position.

\subsection{Source file}

\subsubsection{\sl source.dat \ \ \ }

This ascii file is written with the following format (Object file format -~see
Inputfile~-~):

{\sl int \ \  float1 \ \ \ float2 \ \ \ float3 \ \ \ float4
\ \ \ float5 \ \ \ float6 \ \ \ float7}

n \ \ \ \ x \ \ \ \ \ \ \ \ \ y \ \ \ \ \ \ \ \ \ a \ \ \ \ \ \ \ \ \ 
b \ \ \ \ \ \ \ \ \ $\theta$ \ \ \ \ \ \ \ \ \ z \ \ \ \ \ \ \ \ mag\\

\subsection{Arclet files}

The following  ascii file are written with the following format
(Object file format -~see Inputfile~-~):

{\sl int \ \  float1 \ \ \ float2 \ \ \ float3 \ \ \ float4
\ \ \ float5 \ \ \ float6 \ \ \ float7}

n \ \ \ \ x \ \ \ \ \ \ \ \ \ y \ \ \ \ \ \ \ \ \ a \ \ \ \ \ \ \ \ \ 
b \ \ \ \ \ \ \ \ \ $\theta$ \ \ \ \ \ \ \ \ \ z \ \ \ \ \ \ \ \ \ mag\\
\\


{\sl image.dat \ \ \ } list the images (arclets), but the arclets
with distortion larger than {\bf large{\_}dist } are not included.\\

{\sl image.all \ \ \ } list all images (arclets) with no restriction.\\

{\sl sort.dat \ \ \ } list all images (arclets) sorted from high to low
distortion.

\subsubsection{\sl dist.dat \ \ \ }

This file list some images properties with the following format:

{\sl int \ \  float1 \ \ \ float2 \ \ \ float3 \ \ \ float4
\ \ \ float5 \ \ \ float6 \ \ \ float7 \ \ \ float8}\\

{\sl int \ \ } is the characteristic integer that defines the object.

{\sl float1 \ \ } is the X position of the object expressed in arcsec.

{\sl float2 \ \ } is the Y position of the object expressed in arcsec.

{\sl float3 \ \ } is the distance of the arclet from the center of the first
clump.

{\sl float4 \ \ } is the axis-ratio $b/a$ of the equivalent ellipse of the 
object.

{\sl float5 \ \ } is the ellipticity $\varepsilon$ of the equivalent
ellipse of the object.

{\sl float6 \ \ } is the deformation $\tau$ of the equivalent
ellipse of the object.

{\sl float7 \ \ } is the amplification $\mu$ at the center of the arclet.

{\sl float8 \ \ } is the time-delay $\tau_d$ (in year) 
at the center of the arclet.


\subsubsection{\sl gianti.dat }

File with the list of points that defines the shape of the arc or arclet
when using the  second identifiers {\bf profil} or {\bf contour}.
The format consist of 4 lines of header and then the data:

{\sl float1 float2}\\

{\sl float1 \ \ \ } give the X position of the point.

{\sl float2 \ \ \ } give the Y position of the point.

\subsubsection{\sl "giant"."n"}

Array file of the image of the arc or arclet when using the
second identifiers {\bf iso}. The file consist of a header and then the date.
In the header one can find the number of lines, the number of columns and
the xmin, xmax, ymin, ymax of the surface covered by the array.


\subsection{Critical and caustic lines files}

\subsubsection{\sl ce.dat \ \ \ ci.dat}

These 2 files have the same format, the first one {\sl ce.dat} lists
the external critical and caustic lines ([+,+]-[+,-] transition),
the last one {\sl ci .dat} lists the internal ones ([+,-]-[-,-] transition).\\


{\sl int \ \ \ float1 \ \ \ float2 \ \ \ float3 \ \ \ float 4}\\

{\sl int \ \ \ } enumerates the different lines

{\sl float1 \ \ \ } give the X position of the critical point.

{\sl float2 \ \ \ } give the Y position of the critical point.

{\sl float3 \ \ \ } give the X position of the correspondent caustic point.

{\sl float4 \ \ \ } give the Y position of the correspondent caustic point.


\subsubsection{\sl cr{\_}an.dat \ \ \ }

This file is created if there is only one clump for the mass distribution,
it is an estimate of the critical lines (external and internal). They are
approximated by ellipse, hence they are put in an 'ellipse'-like format:\\

{\sl int \ \  float1 \ \ \ float2 \ \ \ float3 \ \ \ float4
\ \ \ float5 \ \ \ float6}

{\sl int \ \ } non relevant value (1).

{\sl float1 \ \ } is the X position of the center of ellipse expressed
in arcsec.

{\sl float2 \ \ } is the Y position of the center of ellipse expressed
in arcsec.

{\sl float3 \ \ } is the semi-major-axis $a$ of the equivalent ellipse of the 
critical line, expressed in arcsecond.

{\sl float4 \ \ } is the semi-minor-axis $b$ of the equivalent ellipse of the 
critical line, expressed in arcsecond.

{\sl float5 \ \ } is the position-angle $\theta$ of the equivalent
ellipse of the critical line, expressed in degree. This give the
orientation of the semi-major-axis
from the horizontal line (counter-clockwise).

{\sl float6 \ \ } non relevant value (0.).

\vspace{1cm}
The first line is the external critical line, the second line the internal
critical line, both for a Source Plane at redshift {\bf source z\_source}.\\

\subsection{Source marker file}

\subsubsection{\sl marker{\_}s.dat \ \ \ }

Same format as the "Marker file".

\subsection{Prop files}

\subsubsection{\sl "prop".dat \ \ \ }

\subsection{Invert files}

\subsubsection{\sl map.iso \ \ \ }

\subsubsection{\sl map.res \ \ \ }

\subsection{Best file}

\subsubsection{\sl best.par \ \ \ }

\subsubsection{\sl bestopt.par \ \ \ }

\subsection{Bayesian optimisation}
You can read the bayes.dat files with the \textbf{Histogram} and \textbf{Histogram2D} tools. 
They plot 1D and 2D histograms of the samples distribution and give an estimate of the 1 or 2 dimensional marginalised distribution. Those tools have no arguments.


